\documentclass[100pt]{article}
\title{Experiment - 4} % Title
\date{23 feb.2017} % Date for the report

\begin{document}
		\maketitle % Insert the title, author and date
	\section{Aim}
    Uploading the codes on Github.
	\section{Software used}
     Github
	\section{Theory}
    GitHub is a web-based Git or version control repository and Internet hosting service. It offers all of the distributed version control and source code management (SCM) functionality of Git as well as adding its own features. It provides access control and several collaboration features such as bug tracking, feature requests, task management, and wikis for every project.\\
	GitHub offers both plans for private and free repositories on the same account which are commonly used to host open-source software projects. As of April 2016, GitHub reports having more than 14 million users and more than 85.5 million repositories, making it the largest host of source code in the world.\\
	
	 GitHub is mostly used for code.\\
	In addition to source code, \textbf{GitHub supports the following formats and features:}\\
	
  	1.Documentation, including automatically rendered README files in a variety of  Markdown-like file formats\\
	2.Commits history\\
	3.Graphs: pulse, contributors, commits, code frequency, punch card, network\\
	4.Integrations Directory\\
	5.Email notifications\\
	6.GitHub Pages: small websites can be hosted from public repositories on GitHub. The URL format is http://username.github.io.\\
	7.Nested task-lists within files\\
	8.Visualization of geospatial data\\
	9.Photoshop's native PSD format can be previewed and compared to previous versions of the same file.\\
	
	Files that you add to a repository via a browser are limited to 25 MB per file. You can add larger files, up to 100 MB each, via the command line.\\

	\section{Procedure}
   1.On GitHub, navigate to the main page of the repository\\
   
   2.Under your repository name, click Upload files.\\
   
   3.Drag and drop the file or folder you'd like to upload to your repository onto the file tree. \\
   
   4.At the bottom of the page, type a short, meaningful commit message that describes the change you made to the file.\\
   
   5.Below the commit message fields, decide whether to add your commit to the current branch or to a new branch. If your current branch is master, you should choose to create a new branch for your commit and then create a pull request.\\
   
   6.Click Commit changes.\\
   
   \section{Result}
  All MySQL codes are uploaded successfully on Github and the link of github repository is given below - \\
   
   \textbf{https://github.com/mohitnitrr22/mysql-}
\end{document}