\documentclass[12]{article}
\usepackage[margin=1in,paperwidth=8.5in,paperheight=11in]{geometry}
\usepackage{graphicx}
\begin{document}
	\begin{center}
		\begin{Large}
			\textbf{National Institute of Technology, Raipur}\\
			\bigskip
			\textbf{Department of Biomedical Engineering }\\
		\end{Large}
	\end{center}
		\begin{Large}
			\begin{center}
				\textbf{Telemedicine Assignment} \\
				\bigskip
				\textbf{22 April,2017}
			\end{center}
		\end{Large}
		\vspace{5cm}
		 
		\begin{Large}
			\begin{flushleft}
				\textit{Submitted By:}  \hspace{10.2cm} \textit{Guided By:}\\
				\textbf{Name : Mohit Kumar}  \hspace{5.4cm} \textbf{Dr. Sourabh Gupta}\\
				\textbf{Semester :$VI^{th}$}  \hspace{9cm} \\
				\textbf{Roll no : $14111022$ }  
		
			\end{flushleft} 
				\end{Large}
				\newpage
	
		\begin{flushleft}
		\textbf{Challenge 1. Reliable power for medical devices?}\\
		\textbf{Solution:}	
		\end{flushleft}
		 The selection and specification of power supplies for medical applications is a task that must be approached with great care; especially in these times where key safety and environmental standards for medical equipment are undergoing substantial changes that will affect large segments of the medical industry.
		 
		 Modern switchmode power supplies are employed in a wide array of medical equipment including: MRI, X-ray, CT and PET scanners, blood analyzers, DNA equipment, patient monitors, ultrasound, robotic surgical devices, heart-lung machines, diagnostic equipment and automated pharmaceutical dispensers, to name but a few. 
		 
		 Today, medical manufacturers are also introducing portable devices to the marketplace on a daily basis and these applications use rechargeable batteries as the power source.\\
		 VARTA's well-designed battery packs may include a resettable fuse (also known as a polymeric positive temperature coefficient device, or PPTC). \\
		 Several types of protection devices that make the product inherently safer include integrated circuits (ICs) for controlling battery cell voltage. ICs prevent cells from over-charging or discharging by controlling a cut-off switch and monitoring voltage across the switch's MOSFET structures.\\
		 Further design considerations that must be addressed are an understanding of the environment where the device will be used. Hospitals use solvents and disinfectants that could cause corrosive harm to the plastics as well as the electronics inside the battery and that can lead to premature battery failure.\\
		 \begin{flushleft}
		 \hspace{-0.7cm}\textbf{Challenge 2.Long waiting time at primary health care?}\\
		 \hspace{-0.7cm}	\textbf{Solution:}	
		 \end{flushleft}
		 
		 \textbf{1.Gather patient information before their scheduled appointment.\\}
		 
		 This tip may seem like a no-brainer, but there’s always room for improvement. Does your office staff gather insurance information and patient history when they schedule an appointment? Are referrals and patient records always ready and waiting in-office when the patient arrives? Are patients asked to complete and/or send in all necessary forms before their appointment? All of this data and paperwork collection takes time. Allowing patients to fill out any forms on their own time and having all their paperwork ready before the appointment prevents delays at check-in.\\
		 
		 \textbf{2.Delegate Documentation to other trained staff.\\}
		 
		 Whether you’re still adjusting to a new EHR system, or lack the speed of a professional typist, your time should be focused on interacting with patients instead of completing time-consuming documentation. Try implementing a team care model where a clinical assistant takes on some additional documentation tasks like collecting patient history, managing prescription and test orders, and even taking notes during the doctor-patient visit. While adopting this new workflow may take some adjustments and training, it can help prevent you from getting bogged down in more administrative tasks and ensure you’re spending most of your time providing high-quality care to your patients.\\
		 
		 \textbf{3.Use secure messaging.\\}
		 
		 If you use an EHR system, you likely have access to a secure messaging feature that provides an alternative way to communicate important information to your patients. While you may think of secure messaging as “just another thing to manage,” it can actually increase office efficiency and raise patient satisfaction. Dr. Rachel Franklin describes here how secure messaging decreased the number of phone calls her practice received, eliminated “phone tag” or problems with reaching patients at a callback number, and allowed physicians and staff to respond quickly to patient queries. As an added bonus, your patients will love the increased access to their doctor.\\
		 
		 \textbf{4.Create a policy for no-shows and late arrivals and stick to it.\\}.
		 
		 If you haven’t already decided on a policy for dealing with patients who don’t show up or arrive late for their appointments, now’s the time. Set a time limit for late arrivals. If a patient is more than 30 minutes late, let them know you’ll need to reschedule. Charge repeat offenders a cancellation or late fee to motivate them to show up on-time. Make sure to give all patients advance notice of your policies in as many ways as possible (brochures, emails, verbal notice in office and on the phone), and consider giving a free pass and a warning during the grace period. Emphasize that you’re doing this because you value their time as much as yours.\\
		 
		 \textbf{5.Design a survey to identify bottlenecks.\\}
		 
		 Sometimes it’s hard to pinpoint exactly where your daily schedule is running off course. Try handing out a simple survey that tracks each patient’s timeline from arrival to exit. How long are they spending waiting in the reception area or exam room? How long is their visit with the doctor? Remember to use this survey across different days and weeks to get an accurate picture of where the consistent problems are. This method also shows your patients that you value their time.\\
		 
		 \textbf{6.Implement a mobile queue solution.\\}
		 
		 Mobile queue tools are a great way to keep wait times down and patients happy. Applications like Qless give projected wait times and allow patients to let your staff know if they’re running behind. Before they come into a care facility, patients can join a virtual waiting line that updates them on their “position” and enables them to grab lunch, or relax at a nearby coffee shop while they wait. Keeping in constant communication with your patients gives them greater control over their time and helps you manage patient flow.\\
		 
		 \textbf{7.Embrace telehealth solutions.\\}
		 
		 A telehealth solution like eVisit can streamline patient records gathering, prevent no-shows or late arrivals, and cut the average office visit time in half. Plus, virtual treatment options provide convenient in-home physician access to your patients, effectively eliminating time spent traveling to the office or sitting in the waiting room. Telehealth solutions may be the path to a no-wait future care model.\\
		 
		 \textbf{8.Provide a comfortable reception area.\\}
		 
		 Sometimes, even using all of these tricks may not be enough to the keep wait times down. At the very least, make sure your waiting room provides a pleasant space for your patients. Stocking it with magazines and comfy seating, providing complimentary coffee and tea, and offering free wifi or TV entertainment can go a long way in optimizing patient satisfaction even when the wait time isn’t ideal.\\
		 e you value their time as much as yours.\\


		 	\begin{flushleft}
		 		\textbf{Challenge 3.Diarrhea prevention?}\\
		 		\textbf{Solution:}	
		 	\end{flushleft}
		 	The most important way to avoid diarrhea is to avoid coming into contact with infectious agents that can cause it. This means that good hand washing and hygiene are very important.
		 	Also, if you travel to developing countries, you should take the following precautions:
		 	\begin{itemize}
		 		\item Drink only bottled water, even for tooth brushing.
		 		\item Avoid eating food from street vendors.
		 		\item Avoid ice made with tap water.
		 		\item Eat only those fruits or vegetables that are cooked or can be peeled.
		 		\item Be sure that all foods you eat are thoroughly cooked and served steaming hot.
		 		\item Never eat raw or undercooked meat or seafood.
		 		\item Obtain a hepatitis A vaccination prior to travel, if indicated for that region.\\
		 	\end{itemize}
		 	\begin{flushleft}
		    \textbf{Challenge 4.Avaibility of medical gases in remote areas?}\\
		 	\textbf{Ans:}	
		 	\end{flushleft}
		 		\begin{itemize}
		 			\item 	Centralised medical gas pipeline system is a vital and integral part of a modern hospital, with emphasis on safety,	reliability and purity of the gases.
		 			The central piped medical gas system is one of the newer types of hospital plumbing system  introduced into the delievery of direct patient care.
		 			Medical gas piping is needed for oxygen, nitrous oxide,medical air,nitrogen,carbondioxide,vaccum and anesthesia waste exhaust.
		 			Piping gas from the central location directly to outlet at the points of use provides high level of safety unheard of in the past.
		 			These pipe system provide easier quality control and pressure regulation.
		 			\item Transportation of medical gases through closely packed cyllinders in which no leakage is there.
		 		\begin{flushleft}
		 		\textbf{Challenge 5.Improved access:Teleconsultation & Automation?}\\
		 			\textbf{Solution:}	
		 		\end{flushleft}
		 		At a national level, we face a big shortage of trained doctors and paramedics. We only have about 1 doctor for every 1700 patients (WHO stipulates 1:1100). Apart from doctors, we are also short by 64 Lakh Allied Health Professionals. So improvement can be done by teleconsultation and automation.\\
		 		\textbf{Integrated teleconsultation services in cardiology\\}
		 		Integrated regional networks provide an infrastructure for the deployment of accountable, accessible and secure teleconsultation services. In addition, the use of clinical protocols, combined with the automated retrieval of relevant health data, can improve the effectiveness and efficiency of teleconsultation sessions. In the context of HYGEIAnet, the regional health telematics network of Crete, integrated teleconsultation services based on clinical protocols are being developed to support the remote screening of patients with suspected heart problems, aiming not only to benefit the patient but also to contribute to the optimum use of health care resources.\\
		 		\textbf{Web-Based Telehealth Training Platform.\\}
		 		In the interests of patient health outcomes, it is important for medical students to develop clinical communication skills. A telehealth communication skills training platform (EQClinic) with automated nonverbal behavior feedback for medical students, and it was able to improve medical students’ awareness of their nonverbal communication.
		 		
		 	\begin{flushleft}
		 		\textbf{Challenge 6.Assistive technology for Asha worker?}\\
		 		\textbf{Solution:}	
		 	\end{flushleft}
		 	Asha worker stradde the large divide between traditional firms of health care delivery and community centre health care.They are backbones of most government health interventions since they are the "Last mile" of most outreach efforts.\\
		 	Asha worker require assistive technologies for the following activities:
		 	\begin{itemize}
		 		\item Monitoring Patients especially pregnant mothers who require routine follow-up.
		 		\item Reporting the result of their daily work to functionaries of Government of India.
		 		\item Delivering knowledge of better preventive health to patient and their families.
		 		\item Increasing compliance to prescribed medication.
		 	\end{itemize}
		 
\end{document}